\documentclass[12pt]{report}
\usepackage[a4paper, width=150mm, top=25mm, bottom=25mm, headheight=20mm]{geometry}
\usepackage[backend=biber, style=numeric, sorting=none]{biblatex}
\usepackage{graphicx}
\usepackage{fancyhdr}
\usepackage{parskip}
\usepackage{xcolor}
\usepackage{soul}  % For highlighting text - remove when done
\usepackage[hidelinks]{hyperref}

\graphicspath{{./images/}}
\counterwithout{figure}{chapter}
\addbibresource{citations.bib}

\renewcommand{\chaptermark}[1]{\markboth{#1}{}}

% For highlighting text - remove when done
\newcommand{\hlc}[2][yellow]{{%
    \colorlet{foo}{#1}%
    \sethlcolor{foo}\hl{#2}}%
}

\pagestyle{fancy}
\fancyhead[L]{From Scramble to Solution}
\fancyhead[R]{\leftmark}
\fancyfoot[C]{\thepage}

\title{From Scramble to Solution: Developing an Optimal Solver for the Kilominx, a Rubik's Cube Variant}
\author{Samuel Borg}
\date{\today}

\begin{document}

\begin{titlepage}
    \begin{center}
        \vspace*{1.5cm}
            
        \Huge
        \textbf{From Scramble to Solution}
            
        \vspace{0.5cm}
        \LARGE
        Developing an Optimal Solver for \\ the Kilominx, a Rubik's Cube Variant

        \vspace{2cm}
        \textbf{Samuel Borg}

        \Large
        \textit{Supervisor: Ian Gent}

        \vspace{0.5cm}

        \large
        \today

        \vspace{2.5cm}

        \includegraphics[width=0.4\textwidth]{unilogo}

        \Large
        \vspace{0.5cm}
        School of Computer Science\\
        University of St Andrews
    \end{center}
\end{titlepage}

\chapter*{Abstract}
The Rubik's Cube is the most well-known combination puzzle of all time, and has inspired numerous variants of the puzzle. One such puzzle is the Kilominx, a 2x2 dodecahedron-shaped puzzle with 12 faces and 20 cubies (individual cube elements of the puzzle). While the Rubik's Cube has already been extensively researched, the Kilominx has not.

This report details my work in creating an optimal solver for the Kilominx, guaranteeing solutions in as few moves as possible. A separate program I wrote precomputes several pattern databases, each containing the number of moves required to solve different sets of cubies in all possible states. The solver uses an iterative-deepening A* (IDA*) algorithm to find the optimal solution, using the pattern databases as a lower-bound heuristic. In order to efficiently index into the pattern databases, the solver computes a Lehmer code for the puzzle state, which is then used to calculate the state's sequential index.

In theory, the solver can optimally solve any state given enough time, but in practice it can only find solutions up to a depth of 14 within a reasonable timeframe. As solutions require longer sequences of moves to solve, the solver takes exponentially more time to find the optimal solution. With further optimisation, the solver could be used to help tighten the bounds on the Kilominx's God's Number - the maximum number of moves needed to solve the puzzle from any given state.

\chapter*{Declaration}
I declare that the material submitted for assessment is my own work except where credit is explicitly given to others by citation or acknowledgement. This work was performed during the current academic year except where otherwise stated.

The main text of this project report is \hlc[yellow]{[insert word count]} words long, including project specification and plan.

In submitting this project report to the University of St Andrews, I give permission for it to be made available for use in accordance with the regulations of the University Library. I also give permission for the title and abstract to be published and for copies of the report to be made and supplied at cost to any bona fide library or research worker, and to be made available on the World Wide Web. I retain the copyright in this work.

\chapter*{Acknowledgements}
I would like to thank my supervisor, Ian Gent, for his invaluable guidance, insight and support throughout this project. His expertise and enthusiasm have made this project an incredibly engaging and rewarding journey.

I would also like to thank my parents, Juliet and Peter, my brother, Tom, and my amazing friends Emil, Ben, Zoe, Josh, Maya, Jess and Alfie for their unwavering support and belief in me all the way through my studies.

\tableofcontents

\chapter{Introduction}
When the Rubik's Cube was globally released in 1980, it became an instant success, selling around 200 million units by the end of 1983 \cite{unitssold}. The Rubik's Cube also quickly became a popular subject of research for computer scientists, in part due to the massive 43 quintillion possible states \cite{states} that the cube can be in. Many different algorithms were designed to try and solve the cube in as few a number of moves as possible, but it wasn't until 1997 that Richard E. Korf published a paper \cite{korf} describing a method to solve the Rubik's Cube optimally (the shortest possible number of moves to solve any given cube state) by using large lookup tables called pattern databases \cite{patterndatabases} as a heuristic function to guide an IDA* search algorithm. 

Another key area that researchers focused on had to do with a concept called "God's Number", which is the maximum number of moves needed to solve the cube from any given state. Throughout the '90s and '00s, researchers were able to tighten the bounds on the God's Number for the Rubik's Cube, until in 2010 it was finally proved that the cube's God's Number is exactly 20 \cite{godsnumber}. This means that for any optimal solver (such as Korf's algorithm), the solver will always be able to find a solution in 20 moves or less (given enough time to find the solution).

\begin{figure}[h]
    \includegraphics[width=0.4\textwidth]{kilominx}
    \centering
    \caption[An image of a Kilominx.]{An image of a Kilominx\footnotemark.}
\end{figure}

\footnotetext{This image is taken from the \href{https://speedcubesolving.fandom.com/wiki/Kilominx}{"Kilominx"} article on the \href{https://speedcubesolving.fandom.com}{Speed Cube Solving wiki} at \href{https://www.fandom.com/}{Fandom} and is licensed under the \href{https://creativecommons.org/licenses/by-sa/4.0/}{Creative Commons Attribution-Share Alike License}.}

The widespread success of the Rubik's Cube also led to the creation of a number of different variants of the puzzle, each working similarly to the Rubik's Cube but of different shapes and sizes. The Kilominx is one of these variants, part of the larger "minx" family of dodecahedron-shaped puzzles. The Kilominx is a 2x2 dodecahedral puzzle with 12 faces and 20 cubies (individual cube-shaped elements of the puzzle). Although some research has been done on tightening the bounds of its God's Number \cite{kilominxgodsnumber}, no optimal solver has previously been created for the puzzle.

The main goal for this project was to create an optimal solver for the Kilominx, applying similar principles to Korf's algorithm for the Rubik's Cube -  using pattern databases to guide an IDA* search algorithm. An additional goal was to use the solver to help increase the lower bound of Kilominx's God's Number by trying to find an optimal solution of a greater number of moves than the current lower bound.

\section{Project Objectives}
\label{section:objectives}
Below is a list of the objectives for this project, which are grouped based on their priority. It should be noted that the priorities of the objectives are slightly different to the original priorities as outlined in the DOER - these were updated after the project was started to better reflect which objectives were the most important for the project.

\textbf{Primary Objectives:}
\begin{itemize}
    \item Create an optimal solver for the Rubik's Cube, using an IDA* search algorithm and pattern databases (as described in Korf's paper).
    \item Create an optimal solver for the Kilominx, using the same principles as used in the Rubik's Cube solver.
\end{itemize}

\textbf{Secondary Objectives:}
\begin{itemize}
    \item Use the solver to tighten the bounds of the Kilominx's God's Number.
    \item Create a simple GUI to allow users to interact with a Rubik's Cube and Kilominx, and use the solvers to find optimal solutions for both puzzles.
\end{itemize}

\textbf{Tertiary Objectives:}
\begin{itemize}
    \item Survey users about their experience using the GUI and solver to gather feedback on its ease of use.
    \item Investigate better ways to allow users to input a Kilominx state into the solver (e.g. with computer vision).
\end{itemize}

\chapter{Context Survey}
% Surveying the context, the background literature and any
% recent work with similar aims. The context survey describes
% the work already done in this area, either as described in
% textbooks, research papers, or in publicly available software.
% You may also describe potentially useful tools and
% technologies here but do not go into project-specific
% decisions.

\section{Korf's Algorithm for the Rubik's Cube}
\subsection{Iterative-Deepening A*}
\subsection{Pattern Databases}

\section{Sequential Indexing with Lehmer Codes}

\section{God's Number}

\chapter{Requirements Specification}
The following requirements were derived from the primary and secondary project objectives (as outlined in \textbf{\hyperref[section:objectives]{Section 1.1}}).

\textbf{Requirements for Primary Objectives:}
\begin{itemize}
    \item Model the behaviour of a Rubik's Cube and a Kilominx, allowing them to be manipulated by making moves.
    \item Create representations for the pattern databases, which calculate the Lehmer rank of a permutation of cubie indices to determine the permutation's index in the pattern database.
    \item Populate the pattern databases by implementing a breadth-first search (BFS) algorithm or an iterative deepening depth-first search (IDDFS) algorithm.
    \item Implement an IDA* search algorithm to find the optimal solution for a given puzzle state, using the pattern databases as a heuristic function.
    \item Create a command-line interface which lets the user interact with the puzzles and the solver, allowing the user to: view the current state of the puzzle; enter commands to make moves; and use the solver to find an optimal solution for a given puzzle state.
\end{itemize}

\textbf{Requirements for Secondary Objectives:}
\begin{itemize}
    \item Use the solver to increase the lower bound of the Kilominx's God's Number by finding an optimal solution of a greater number of moves than the current lower bound.
    \item Create a simple GUI program which uses the previously created classes to allow users to interact with a Rubik's Cube and Kilominx.
    \item Connect the solver to the GUI to find the optimal solution for the puzzle, and then display the solution to the user.
\end{itemize}

% \chapter{Software Engineering Process}
% % The development approach taken and justification for its
% % adoption.

% \chapter{Ethics}
% % Any ethical considerations for the project. You should scan
% % the signed ethical approval document, and include it as an
% % appendix. 

\chapter{Design}
\input{chapters/4-design}

\chapter{Implementation}
\input{chapters/5-implementation}

\chapter{Evaluation}
\input{chapters/6-evaluation}

\chapter{Conclusions}
\input{chapters/7-conclusions}

\appendix
\chapter{Testing Summary}
% This should describe the steps taken to debug, test, verify or
% otherwise confirm the correctness of the various modules
% and their combination.

\chapter{User Manual}
% Instructions on installing, executing and using the system
% where appropriate.

\printbibliography[heading=bibintoc, title={References}]

\end{document}
