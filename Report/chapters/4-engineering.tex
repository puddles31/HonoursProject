\section{Software Development Process}
An incremental development process was used for this project in order to ensure steady progress throughout both semesters. This approach allowed me to slowly build up the solver in small, functional increments, where each increment was thoroughly tested and documented before moving on to the next. This method helped ensure that the final solution was robust and well-documented.

At the start of the project, I began by creating a reference guide for creating an optimal solver for the Rubik's Cube. This guide detailed each of the steps required to model the Rubik's Cube and then create the optimal solver for it, highlighting any important design decisions that needed to be made along with concise justifications for the decisions. This guide was used as a roadmap for the development of the solver, ensuring that I stayed on track towards the final goal.

After creating the reference guide, I followed it to implement an optimal solver for the Rubik's Cube. While creating a solver for the Rubik's Cube was not part of the project objectives, it was a necessary step in order to gain a better understanding of the process required to create an optimal solver. I took care while implementing the classes and methods to ensure that they were modular and reusable - this allowed me to easily extend the both the solver and other classes to work with the Kilominx later on. The Rubik's Cube solver was thoroughly tested to ensure that it was working correctly before moving on to implementing the Kilominx solver.

Finally, I implemented the Kilominx solver following the same steps from the reference guide as I did for the Rubik's Cube solver. I reused many of the classes and methods from the Rubik's Cube solver, extending them where necessary to work with the Kilominx. The Kilominx solver was also thoroughly tested to ensure that it was working correctly.

\section{Choice of Language}
I decided to use Java as the primary programming language for the solver. This decision was made based on my familiarity with Java and its object-oriented principles, such as abstract classes and inheritance, which I believed would be helpful for allowing parts of the code to be reused and extended. Additionally, Java is a relatively fast language and is well-suited for the type of computation required for solving the Rubik's Cube and Kilominx. There was some consideration given to using C++ due to its speed and efficiency, but due to my lack of experience with the language, I ultimately decided that Java would be the best choice for this project.
